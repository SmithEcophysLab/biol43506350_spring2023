\documentclass[12pt, notitlepage]{article}   	% use "amsart" instead of "article" for AMSLaTeX format
\usepackage{geometry}                		% See geometry.pdf to learn the layout options. There are lots.
\geometry{a4paper}                   		% ... or a4paper or a5paper or ... 
%\geometry{landscape}                		% Activate for rotated page geometry
\usepackage[parfill]{parskip}    		% Activate to begin paragraphs with an empty line rather than an indent
\usepackage{graphicx}				% Use pdf, png, jpg, or eps§ with pdflatex; use eps in DVI mode
								% TeX will automatically convert eps --> pdf in pdflatex

\usepackage{hyperref}


%SetFonts

\usepackage[T1]{fontenc}
\usepackage[utf8]{inputenc}

\usepackage{tgbonum}

%SetFonts

\title{
	\textbf{
		Mini-Quiz 6
	} \\
	\large BIOL 4350/6350 \\
	\large Ecophysiology \\
}

\date{\vspace{-5ex}}

\def\wl{\par \vspace{\baselineskip}}

\begin{document}

{\fontfamily{phv}\selectfont %select helvetica (code = phv)

\maketitle

\section{\small{How would you expect elevated CO2 to change the competition between C3 
and C4 species? Why?}}

\newpage

\section{\small{Imagine your favorite ecosystem (natural or anthropogenic). 
Briefly introduce it it and use a systems diagram to explain how you would expect total carbon uptake
to differ with elevated CO2? Use boxes and arrows for your diagram and add either a "+", "-", or "+/-"
sign to each arrow. Your left-hand box should be "atmospheric CO2" and your right-hand box should
be "ecosystem CO2 uptake".}}

\wl
\wl
\wl
\wl
\wl
\wl
\wl
\wl
\wl
\wl
\wl
\wl
\wl
\wl


} %end font selection

\end{document}