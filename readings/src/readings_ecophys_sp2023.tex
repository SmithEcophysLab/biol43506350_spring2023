\documentclass[12pt, notitlepage]{article}   	% use "amsart" instead of "article" for AMSLaTeX format
\usepackage{geometry}                		% See geometry.pdf to learn the layout options. There are lots.
\geometry{a4paper}                   		% ... or a4paper or a5paper or ... 
%\geometry{landscape}                		% Activate for rotated page geometry
\usepackage[parfill]{parskip}    		% Activate to begin paragraphs with an empty line rather than an indent
\usepackage{graphicx}				% Use pdf, png, jpg, or eps§ with pdflatex; use eps in DVI mode
								% TeX will automatically convert eps --> pdf in pdflatex

\usepackage{hyperref}


%SetFonts

\usepackage[T1]{fontenc}
\usepackage[utf8]{inputenc}

\usepackage{tgbonum}

%SetFonts

\title{
	\textbf{
		Readings
	} \\
	\large Plant Physiological Ecology \\
	\large Spring 2021
}

\date{\vspace{-5ex}}

\begin{document}

{\fontfamily{phv}\selectfont %select helvetica (code = phv)

\maketitle

**Please contact Dr. Smith if you have trouble accessing the articles**

**Note: this file will be updated to account for changes to the schedule**

\section*{Week of January 16}
\textit{Classical Literature Tuesday - Jan 17} \par
Chapin FS. 2003. Effects of Plant Traits on Ecosystem and Regional Processes: 
a Conceptual Framework for Predicting the Consequences of Global Change. 
Annals of Botany 91: 455–463. \par
\url{https://academic.oup.com/aob/article/91/4/455/213070}

\textit{Recent Literature Thursday - Jan 19} \par
Reich PB. 2014. The world-wide ‘fast–slow’ plant economics spectrum: a traits manifesto. 
Journal of Ecology 102: 275–301. \par
\url{https://besjournals.onlinelibrary.wiley.com/doi/10.1111/1365-2745.12211}
\par

\section*{Week of January 23}
\textit{Classical Literature Tuesday - Jan 24} \par
Von Caemmerer S, Farquhar GD. 1981. Some relationships between the biochemistry of 
photosynthesis and the gas exchange of leaves. Planta 153: 376–387. \par
\url{https://link.springer.com/article/10.1007/bf00384257}

\textit{Recent Literature Thursday - Jan 26} \par
Wang, Z., Wang, C., & Liu, S. (2022). Elevated CO2 alleviates adverse effects of drought 
on plant water relations and photosynthesis: A global meta‐analysis. Journal of 
Ecology, 110(12), 2836-2849. \par
\url{https://besjournals.onlinelibrary.wiley.com/doi/full/10.1111/1365-2745.13988}

\section*{Week of January 30}
\textit{Classical Literature Tuesday - Jan 31} \par
Boardman NK. 1977. Comparative photosynthesis of sun and shade plants. 
Annual review of plant physiology 28: 355–377. \par
\url{https://www.annualreviews.org/doi/10.1146/annurev.pp.28.060177.002035}

\textit{Recent Literature Thursday - Feb 2} \par
Paik I, Huq E. 2019. Plant photoreceptors: Multi-functional sensory proteins and 
their signaling networks. Seminars in Cell & Developmental Biology 92, 114–121.\par
\url{https://www.sciencedirect.com/science/article/pii/S1084952117305748}

\section*{Week of February 6}
\textit{Classical Literature Tuesday - Feb 7} \par
Atkin OK and Tjoelker M. 2003. Thermal acclimation and the dynamic response of plant 
respiration to temperature. Trends in Plant Science 8: 343–351. \par
\url{https://www.sciencedirect.com/science/article/pii/S1360138503001365}

\textit{Recent Literature Thursday - Feb 9} \par
Posch BC, Kariyawasam BC, Bramley H, Coast O, Richards RA, Reynolds MP, Trethowan R, 
Atkin OK. 2019. Exploring high temperature responses of photosynthesis and respiration 
to improve heat tolerance in wheat. Journal of Experimental Botany 70, 5051–5069. \par
\url{https://academic.oup.com/jxb/article/70/19/5051/5506706}

\section*{Week of February 13}
\textit{Classical Literature Tuesday - Feb 14} \par
Chaves MM, Pereira JS, Maroco J, et al. 2002. How Plants Cope with Water Stress 
in the Field? Photosynthesis and Growth. Annals of Botany 89: 907–916. \par
\url{https://academic.oup.com/aob/article/89/7/907/151103}

\textit{Recent Literature Thursday - Feb 16} \par
Zhao J, Feng H, Xu T, Xiao J, Guerrieri R, Liu S, Wu X, He X, He X. 2021. 
Physiological and environmental control on ecosystem water use efficiency 
in response to drought across the northern hemisphere. Science of The Total Environment 758, 143599. \par
\url{https://www.sciencedirect.com/science/article/pii/S0048969720371308}

\section*{Week of February 20}
NO CLASS

\section*{Week of February 27}
\textit{Classical Literature Tuesday - Feb 28} \par
Bazzaz FA. 1990. The response of natural ecosystems to the rising global CO2 levels. 
Annual review of ecology and systematics 21: 167–196. \par
\url{https://www.annualreviews.org/doi/10.1146/annurev.es.21.110190.001123}

\textit{Recent Literature Thursday - Mar 2} \par
TBD \par
\url{TBD}

\section*{Week of March 6}
\textit{Classical Literature Tuesday - Mar 7} \par
LeBauer, D. S. and Treseder, K. K. (2008), Nitrogen limitation of net primary productivity
in terrestrial ecosystems is globally distributed. Ecology, 89: 371-379. \par
\url{https://esajournals.onlinelibrary.wiley.com/doi/full/10.1890/06-2057.1}

\textit{Recent Literature Thursday - Mar 8} \par
TBD \par
\url{TBD}

\section*{Week of March 13}
NO CLASS.

\section*{Week of March 20}
\textit{Classical Literature Tuesday - Mar 21} \par
Mooney HA. 1972. The carbon balance of plants. 
Annual review of ecology and systematics 3: 315–346. \par
\url{https://www.annualreviews.org/doi/10.1146/annurev.es.03.110172.001531}

\textit{Recent Literature Thursday - Mar 23} \par
TBD \par
\url{TBD}

\section*{Week of March 27}
PRESENTATION WEEK. NO READINGS.

\section*{Week of April 3}
NO CLASS.

\section*{Week of April 10}
\textit{Classical Literature Tuesday - Apr 11} \par
Grime JP. 1977. Evidence for the Existence of Three Primary Strategies in Plants and Its 
Relevance to Ecological and Evolutionary Theory. 
The American Naturalist 111: 1169–1194. \par
\url{https://www.jstor.org/stable/2460262}

\section*{Week of April 17}
PRESENTATION WEEK. NO READINGS.

\section*{Week of April 24}
NO CLASS.

\section*{Week of May 1}
PRESENTATION WEEK. NO READINGS.

} %end font selection

\end{document}